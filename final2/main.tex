\documentclass[a4j,12pt]{jreport}
\usepackage[dvips]{graphicx}
\usepackage{mythesis}
\usepackage{multirow}
\usepackage{here}
\setlength{\itemsep}{-1zh}
\title{}
\icon{
		\includegraphics[width=80mm,bb=0 0 595 842]{fig/ryukyu.pdf}
	}
\year{平成Y年度 卒業論文}
\belongto{琉球大学工学部情報工学科}
\author{0957  \\ 指導教員 {} }
%%
%% プリアンブルに記述
%% Figure 環境中で Table 環境の見出しを表示・カウンタの操作に必要
%%
\makeatletter
\newcommand{\figcaption}[1]{\def\@captype{figure}\caption{#1}}
\newcommand{\tblcaption}[1]{\def\@captype{table}\caption{#1}}
\makeatother
\setlength\abovecaptionskip{0pt}

\begin{document}

% タイトル
\maketitle
\baselineskip 17pt plus 1pt minus 1pt

\pagenumbering{roman}
\setcounter{page}{0}

\tableofcontents	% 目次
\listoffigures		% 図目次
\listoftables		% 表目次

%以下のように、章ごとに個別の tex ファイルを作成して、
% main.tex をコンパイルして確認する。
%章分けは個人で違うので下のフォーマットを参考にして下さい。

% はじめに
\chapter{はじめに}
\label{chap:introduction}
\pagenumbering{arabic}

%序論の目安としては1枚半ぐらい.
%英語発表者は,最終予稿の「はじめに」の英訳などを載せてもいいかも.

\section{背景と目的}


\section{論文の構成}

%\section{Introduction}


% 基礎概念
\chapter{基礎概念}
\label{chap:concept}

\section{}


\section{}


% 実験
\chapter{実験}
\label{chap:poordirection}


\section{実験説明}

\section{}

\section{検証結果}



\section{考察}


% 他の論文との比較
%\input{chapter4.tex}

% 今後の課題
\chapter{今後の課題}


% 参考文献
% 参考文献
\def\line{−\hspace*{-.7zw}−}

\begin{thebibliography}{99}
%\bibitem{*}内の * は各自わかりやすい名前などをつけて、
%論文中には \cite{*} のように使用する。
%これをベースに書き換えた方が楽かも。
%書籍、論文、URLによって若干書き方が異なる。
%URLを載せる人は参考にした年月日を最後に記入すること。


\bibitem{hoge}
hoge
\end{thebibliography}


% 謝辞
\chapter*{謝辞}
\thispagestyle{empty}

%基本的な内容は以下の通り.参考にしてみて下さい.
%厳密な決まりは無いので,個々人の文体でも構わない.
%GISゼミや英語ゼミに参加した人はその分も入れておく.
%順番は重要なので気を付けるように.(提出前に周りの人に確認してもらう.)

\hspace{1zw}本研究の遂行,また本論文の作成にあたり、御多忙にも関わらず終始懇切なる御指導と御教授を賜わりましたhoge助教授に深く感謝したします。

また、本研究の遂行及び本論文の作成にあたり、日頃より終始懇切なる御教授と御指導を賜わりましたhoge教授に心より深く感謝致します。

数々の貴重な御助言と細かな御配慮を戴いたhoge研究室のhoge氏に深く感謝致します。

また一年間共に研究を行い、暖かな気遣いと励ましをもって支えてくれたhoge研究室のhoge君、hoge君、hogeさん並びにhoge研究室のhoge、hoge君、hoge君、hoge君、hoge君に感謝致します。

最後に、有意義な時間を共に過ごした情報工学科の学友、並びに物心両面で支えてくれた両親に深く感謝致します。

\begin{flushright}
 2010年 3月 \\ hoge
\end{flushright}




% 付録
%\input{appendix.tex}

\end{document}
